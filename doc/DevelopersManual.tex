% Use the OpenCTM TeX style
% Document class
\documentclass[11pt,a4paper]{report}

% Packages
\usepackage[latin1]{inputenc}
\usepackage{amsmath}
\usepackage{amsfonts}
\usepackage{amssymb}
\usepackage{listings}
\usepackage{color}
\usepackage{times}

% Paragraph styles
\raggedright
\usepackage{parskip}

% Code listings
\lstset{frame=single,frameround=tttt,backgroundcolor=\color{code},%
        language=C,basicstyle={\ttfamily\small},%
        breaklines,breakindent=0pt,postbreak=\space\space\space\space}

% Colors
\definecolor{link}{rgb}{0.6,0.0,0.0}
\definecolor{code}{rgb}{0.9,0.9,1.0}
\definecolor{codeA}{rgb}{0.9,1.0,0.9}
\definecolor{codeB}{rgb}{1.0,0.9,0.9}

% PDF specific document properties
% hyperref (bookmarks, links etc) - use this package last
\usepackage[colorlinks=true,linkcolor=link,bookmarks=true,bookmarksopen=true,%
            pdfhighlight=/N,bookmarksnumbered=true,bookmarksopenlevel=1,%
            pdfview=FitH,pdfstartview=FitH]{hyperref}
\hypersetup{pdftitle={OpenCTM Developers Manual}}
\hypersetup{pdfauthor={Marcus Geelnard}}
\hypersetup{pdfkeywords={OpenCTM,manual}}


% Document properties
\author{Marcus Geelnard}
\title{OpenCTM Developers Manual}

% Document contents
\begin{document}

\maketitle

\tableofcontents

\chapter{Introduction}
The OpenCTM file format is an open format for storing 3D triangle meshes.
One of the main advantages over other similar file formats is its ability
to losslessly compress the triangle geometry to a fraction of the corresponding
raw data size.

\chapter{Installation}

\section{Windows - MS Visual Studio 2008}
\subsection{Build from source}
Start an MS Visual Studio command line (for instance using the Tools $>$
Visual Studio 2008 Command Prompt menu from the IDE), and do the following:

\begin{lstlisting}
> cd [sourcedir]
> nmake /fMakefile.msvc
\end{lstlisting}

...where [sourcedir] is the location of the OpenCTM source. This should produce the
files openctm.dll, openctm.lib, tools$\backslash$ctmconv.exe and
tools$\backslash$ctmviewer.exe (the latter requires GLUT).

\subsection{C/C++ link library}
The LIB file, openctm.lib, should be placed in the Visual Studio link library
directory.

\subsection{Dynamic link library (DLL)}
The DLL file, openctm.dll, should be placed in the same directory as your application executable.


\section{Windows - MinGW32}


\section{Mac OS X}


\section{Linux}


\chapter{Usage}
Bla-bla-bla...

\chapter{API Reference}
Bla-bla-bla...

\section{ctmCreateContext}
\begin{lstlisting}
CTMcontext ctmCreateContext(CTMenum aMode)
\end{lstlisting}

\end{document}
